% % % General report document template
% % % based upon the homework template by Ted Pavlic
% % % http://www.tedpavlic.com/post_homework_tex_example.php
% % % Author: Ernest Skrzypczyk

% % % Document properties
% % Document class
\documentclass{report}
%\documentclass{article}
% "article" = page number on title page
% "report" = no page number on title page

% % % Naming conventions
% % Cross references
% Algorithms - \label{a:label} \aref{a:label}
% Equations - \label{e:label} \eref{e:label}
% Figures - \label{f:label} \fref{f:label}
% Sections - \label{s:label} \sref{s:label}
% Scripts - \label{r:label} \rref{r:label}
% Tables - \label{t:label} \tref{t:label}
% Subsections - \label{s:mainlabel--subsectionlabel} \ref{s:mainlabel--subsectionlabel}
% Subsubsections - \label{s:mainlabel--subsectionlabel--subsubsectionlabel} \ref{s:mainlabel--subsectionlabel--subsubsectionlabel}
% % Numbering
% Sections - n * 10; n = 1, 2, ... - gives the possibility to shift other sections in between
% Equations - Section label ["--" subsection label] "--" abbreviated name
% Figures - Section label ["--" subsection label] "--" abbreviated name

% % % User packages
% % Languages
\usepackage{algorithm}
\usepackage{algorithmic}
\usepackage{amsmath,amsfonts,amsthm,amssymb}
\usepackage{array}
%\usepackage[T1]{fontenc}
%\usepackage[MeX]{polski}
\usepackage{polski}
%\usepackage[utf8]{inputenc}
\usepackage[polish,german,english]{babel}
% % General
\usepackage[usenames,dvipsnames]{color}
%\usepackage{cleveref}
\usepackage{calc}
\usepackage{caption}
\usepackage{chngcntr}
\usepackage{extramarks}
\usepackage{enumitem}
\usepackage{fancyhdr}
\usepackage{float}
\usepackage{fontspec}
\usepackage{gensymb}
\usepackage{graphicx}
\usepackage{hyperref}
\usepackage{chngpage}
\usepackage{soul}
\usepackage[retainorgcmds]{IEEEtrantools}
\usepackage{ifthen}
\usepackage{lastpage}
\usepackage{listings}
\usepackage{cancel}
\usepackage{courier}
\usepackage{layout}
\usepackage{lscape}
\usepackage{makecell}
\usepackage{media9}
\usepackage{multicol}
\usepackage{multimedia}
\usepackage{multirow}
\usepackage[numbers]{natbib}
\usepackage{pdflscape}
\usepackage{pdfpages}
\usepackage{setspace}
%\usepackage{showframe}
\usepackage{subcaption}
%\usepackage{subfigure}
\usepackage{tabularx}
\usepackage{titlesec}
\usepackage{varioref}
\usepackage{wrapfig}
%\usepackage[table]{xcolor}
\usepackage{xfrac}
\usepackage{xstring}
\usepackage{tikz}


% % % Packages configuration
% % Tabular
\newcolumntype{Y}{>{\centering\arraybackslash}X}
%\newcolumntype{C}{>{\centering\arraybackslash}m{0.1\textwidth}}
%\newcolumntype{C}{>{\centering\arraybackslash}m{\dimexpr.25\linewidth-2\tabcolsep}}
\newcolumntype{C}{>{\centering\arraybackslash}m{\dimexpr.245\linewidth-2\tabcolsep}}
%\counterwithin{figure}{section}


% % % Command definitions
% % References
\newcommand{\aref}[1]{\autoref{#1}}
\newcommand{\bref}[1]{(\ref{#1})}
%\newcommand{\cref}[1]{(\afig\ref{#1})}
%\newcommand{\eref}[1]{(E\ref{#1})}
%\newcommand{\fref}[1]{(F\ref{#1})}
%\newcommand{\rref}[1]{(S\ref{#1})}
\newcommand{\eref}[1]{(\ref{#1})}
\newcommand{\fref}[1]{(\ref{#1})}
\newcommand{\rref}[1]{(\ref{#1})}
%\newcommand{\sref}[1]{\emph{(\autopageref{#1}) -- \nameref{#1}}}
\newcommand{\sref}[1]{\emph{"\nameref{#1}"}}
\newcommand{\tref}[1]{\ref{#1}}
% % Symbols
\newcommand{\dlsh}{\reflectbox{\rotatebox[origin=c]{180}{$\Lsh$}}}
\newcommand{\drsh}{\reflectbox{\rotatebox[origin=c]{180}{$\Rsh$}}}
\newcommand \ccancel[2][black]{\renewcommand \CancelColor{\color{#1}}\cancel{#2}}
\newcommand \ccancelto[3][black]{\renewcommand \CancelColor{\color{#1}}\cancelto{#2}{#3}}
% % Font styles
\newcommand \dunderline[1]{\underline{\underline{\ #1\ }}}
\newcommand \sunderline[1]{\underline{\ #1\ }}
% % Mathematical symbols and structures
\renewcommand{\vec}[1]{\overrightarrow{\mathbf{#1}}}
%\renewcommand{\vece}[1]{\overrightarrow{\mathbf{#1}}}
\newcommand{\vecg}[1]{\underline{\mathbf{#1}}}
%\renewcommand{\vec}[1]{\mathbf{\overline{#1}}}
\newcommand{\ssl}[1]{\phantom{}^{#1}}
% % Other
\StrLeft{\figurename}{1}[\afig]


% % % Configuration
\graphicspath{{./}{Graphics/}}
%\renewcommand{\figurename}{Wykres}
%\renewcommand{\figurename}{Figure}
%\renewcommand{\figurename}{Image}
%\renewcommand{\tablename}{Tabela}
%\renewcommand{\lstlistingname}{Skrypt}
\renewcommand{\lstlistingname}{Script}
\newcommand{\lstlistingnameexcerpt}{Script excerpt}


% % Complex definitions
% Links
% Generic hyperlink -- \link[optional text]{link}
\newcommand*{\link}[2][]{
	\ifthenelse{\equal{#1}{}}
	{\href{#2}{#2}}
	{\href{#2}{#1}}}

% Http hyperlink -- \httplink[optional text]{link}
\newcommand*{\httplink}[2][]{
	\ifthenelse{\equal{#1}{}}
	{\href{http://#2}{#2}}
	{\href{http://#2}{#1}}}

% Email hyperlink -- \emaillink[optional text]{link}
\newcommand*{\emaillink}[2][]{
	\ifthenelse{\equal{#1}{}}
	{\href{mailto:#2}{#2}}
	{\href{mailto:#2}{#1}}}

% Graphics
% Inkscape path
%\IfFileExists{/dev/null}{
%	\newcommand{\Inkscape}{/usr/bin/inkscape }}{
%	\newcommand{\Inkscape}{"C:/Program Files (x86)/Inkscape/inkscape.exe" }
%}

% Include SVG-LaTeX within scalebox
%\includesvga[0.75]{Graphics/10-odometry--right-turn}
%\includesvga[scalebox]{input-files-w/o-ext}
\newcommand{\includesvgsb}[2][1]{
%	\IfFileExists{\Inkscape}{
%		\immediate\write18{\Inkscape -z -D --file="#2.svg" --export-pdf="#2.pdf" --export-dpi=72 --export-latex}}{}
	\scalebox{#1}{\input{#2.pdf_tex}}
}

% Include SVG-LaTeX as figure
%\includesvg[0.75]{a}{Automated includesvg pdf\label{f:f4b}}
%\includesvg[scale]{input-files-w/o-ext}{caption\label{f:id}}
\newcommand{\includesvg}[3][1]{
%	\IfFileExists{\Inkscape}{
%		\immediate\write18{/home/emeres/.bin/testbell}
%		\immediate\write18{\Inkscape -z -D --file="#2.svg" --export-pdf="#2.pdf" --export-dpi=96 --export-latex}}{}
	\begin{figure}[!ht]
%		\centering
		\begin{center}
			\def\svgwidth{#1\columnwidth}
			\input{#2.pdf_tex}
			\caption{#3}
		\end{center}
	\end{figure}
}

% Scaled figure
\newcommand{\scalefig}[3]{
	\begin{figure}[!ht]
%		\centering
		\begin{center}
			\includegraphics[width=#2\columnwidth]{#1}
			\caption{#3}
		\end{center}
	\end{figure}
}

% Scaled and trimmed figure
\newcommand{\scalefigcrop}[7]{
	\begin{figure}[!ht]
%		\centering
		\begin{center}
			\includegraphics[trim=#4 #5 #6 #7, clip, width=#2\columnwidth]{#1}
			\caption{#3}
		\end{center}
	\end{figure}
}

% Function reference
\newcommand{\func}[2]
{\link[\texttt{#1}]{#2}}

% Varaible reference
\newcommand{\var}[2]
{\link[\texttt{#1}]{#2}}

% File reference
\newcommand{\file}[2]
{\link[\texttt{#1}]{file:///#2}}

% Script listing
%\newcommand{\script}[3]
%{\lstinputlisting[caption=#2,label=#1]{#1.#3}}
\newcommand{\script}[2]
{\lstinputlisting[caption=#2]{#1}}

\lstloadlanguages{MATLAB}
\lstset{language=MATLAB,
	frame=single,	% Frame type around code
	basicstyle=\small\ttfamily,	% Basic code style
	keywordstyle=[1]\color{Blue}\bf,	% Python functions bold and blue
	keywordstyle=[2]\color{Purple},	% Python function arguments purple
	keywordstyle=[3]\color{Blue}\underbar,	% User functions underlined and blue
	identifierstyle=,	% Nothing special about identifiers
	commentstyle=\usefont{T1}{pcr}{m}{sl}\color{DarkGreen}\small,	% Comment style
	stringstyle=\color{Purple},	% String style
	showstringspaces=false,	% Space string indication
	tabsize=2,
%	escapechar={},
	breaklines=true,
	breakautoindent=false,
	prebreak=\raisebox{0ex}[0ex][0ex]{\ensuremath{\space\color{red}{\blacktriangledown}\color{blue}\thelstnumber}},
	postbreak=\raisebox{0ex}[0ex][0ex]{\ensuremath{\color{blue}\thelstnumber\color{red}{\blacktriangle}\space}},
	morekeywords={xlim, ylim, },	% Not default functions
	morekeywords=[2]{on, off},	% Custom function parameters
	morekeywords=[3]{FindESS},	% Custom functions
	morecomment=[l][\color{Blue}]{...},	% Line continuation (...) like blue comment
	numbers=left,
	numberfirstline=true,
	firstnumber=1,	% Line number offset
	numberstyle=\tiny\color{Blue},	% Line numbers
	stepnumber=5	% Line numbers step
}

\newcommand{\scriptexcerpt}[2]{
	\newcommand{\lstlistingnametmp}{\lstlistingname}
	\renewcommand{\lstlistingname}{\lstlistingnameexcerpt}
	\lstinputlisting[caption=#2]{#1}
	\renewcommand{\lstlistingname}{\lstlistingnametmp}
}


% % % Specific Information
% % Course related
\newcommand{\Code}{RM}
%\newcommand{\Title}{Research Methodology}
%\newcommand{\SubTitle}{Bibliography - UGV and UAV collaboration in an autonomous infrastructure scenario}
\newcommand{\Title}{UGV and UAV collaboration in an autonomous infrastructure scenario}
\newcommand{\STitle}{UGV \& UAV collaboration within autonomous infrastructure}
\newcommand{\SubTitle}{Bibliography}
\newcommand{\Date}{24.01.17}
\newcommand{\Time}{Wed 23:59}
\newcommand{\Recipient}{Prof. I. Taralova, Prof. O. Kermorgant}
\newcommand{\Author}{Ernest Skrzypczyk, BSc}

% % PDF general
\hypersetup{
	pdftitle = {\Title \ -- \SubTitle},
	pdfauthor = {\Author},
	pdfkeywords = {robotics, collaboration, slam, multislam, autonomous, infrastructure, unmanned, vehicle, uav, ugv}
}

% % % Document margins
\topmargin = -0.45in
\evensidemargin = 0in
\oddsidemargin = 0in
\textwidth = 6.5in
\textheight = 9.0in
\headsep = 0.25in


% % % Setup the header and footer
\pagestyle{fancy}
\lhead{\Author}
\chead{\Code: \STitle, \Recipient\\} %(\Time,\ \Date)} %
\rhead{\firstxmark}
\lfoot{\lastxmark}
\cfoot{}
\rfoot{Page\ \thepage/\protect\pageref{LastPage}}
\renewcommand \headrulewidth{0.4pt}
\renewcommand \footrulewidth{0.4pt}
%\setlength{\parindent}{4ex}
\setlength\parindent{24pt}

% % % Colors
\definecolor{DarkGreen}{rgb}{0.0,0.4,0.0}
\IfFileExists{Graphics/color-palette.tex}{\input{Graphics/color-palette.tex}}

% % % Debugging
%\tracingall


% % % % CLEAN UP
% % % Style specific commands and configuration
%\newcommand{\enterProblemHeader}[1]{\nobreak\extramarks{#1}{#1 kontynuowany na następnej stroniecontinued on next page\ldots}\nobreak%
%\nobreak\extramarks{#1 (continued)}{#1 continued on next page\ldots}\nobreak}
%\newcommand{\exitProblemHeader}[1]{\nobreak\extramarks{#1 (continued)}{#1 continued on next page\ldots}\nobreak%
%\Title\ifthenelse{\equal{\SubTitle}{}}{}{\\\SubTitle}}}
\newcommand{\enterProblemHeader}[1]{\nobreak\extramarks{#1}{#1\ldots}\nobreak%
	\nobreak\extramarks{#1 (continued)}{#1}\nobreak}
\newcommand{\exitProblemHeader}[1]{\nobreak\extramarks{#1 (continued)}{#1}\nobreak%
	\nobreak\extramarks{#1}{}\nobreak}

\newlength{\labelLength}
\newcommand{\labelAnswer}[2]
	{\settowidth{\labelLength}{#1}
	\addtolength{\labelLength}{0.25in}
	\changetext{}{-\labelLength}{}{}{}
	\noindent\fbox{\begin{minipage}[c]{\columnwidth}#2\end{minipage}}
	\marginpar{\fbox{#1}}

	% We put the blank space above in order to make sure this
	% \marginpar gets correctly placed.
	\changetext{}{+\labelLength}{}{}{}}

\setcounter{secnumdepth}{0}
\newcommand{\ProblemName}{}
\newcounter{ProblemCounter}
\newenvironment{Problem}[1][Problem \arabic{ProblemCounter}]%
	{\stepcounter{ProblemCounter}
	\renewcommand{\ProblemName}{#1}
%	\section{\ProblemName}
	\enterProblemHeader{\ProblemName}}
	{\exitProblemHeader{\ProblemName}}

\newcommand{\problemAnswer}[1]
	{\noindent\fbox{\begin{minipage}[c]{\columnwidth}#1\end{minipage}}}

\newcommand{\problemLAnswer}[1]
	{\labelAnswer{\ProblemName}{#1}}

\newcommand{\SectionName}{}
\newlength{\SectionLabelLength}{}
\newenvironment{Section}[1]
	{\renewcommand{\SectionName}{#1}
	\settowidth{\SectionLabelLength}{\SectionName}
	\addtolength{\SectionLabelLength}{0.25in}
	\changetext{}{-\SectionLabelLength}{}{}{}
%	\subsection{\SectionName}
	\enterProblemHeader{\ProblemName\ [\SectionName]}}

{\enterProblemHeader{\ProblemName} \changetext{}{+\SectionLabelLength}{}{}{}}
%FIX Determine total character length, substract from it the authors string and limit problemname string accordingly, in worst case skip it or use alternative name
\newcommand{\sectionAnswer}[1]
{
	\noindent\fbox{\begin{minipage}[c]{\columnwidth}#1\end{minipage}}
	\enterProblemHeader{\ProblemName}\exitProblemHeader{\ProblemName}
	\marginpar{\fbox{\SectionName}}
}

%% Edits the caption width
%\newcommand{\captionwidth}[1]{%
% \dimen0=\columnwidth	\advance\dimen0 by-#1\relax
% \divide\dimen0 by2
% \advance\leftskip by\dimen0
% \advance\rightskip by\dimen0
%}
% % % % CLEAN UP


% % % % CLEAN UP
% % % Title page
\title{\vspace{2in}\textmd{\textbf{\Code:\ 
\Title\ifthenelse{\equal{\SubTitle}{}}{}{\\\SubTitle}}} \\\normalsize\vspace{0.1in}\small{Date:\ 
\Time,\ \Date} \\\vspace{0.1in}\large{\textit{\Recipient}}\vspace{3in}}
\date{}
\author{\textbf{\Author}}
% % % % CLEAN UP


% % % Document
\begin{document}
\begin{spacing}{1.1}
%\maketitle
\input{Contents/00-titlepage.tex}
%\includegraphics{Contents/00-titlepage.pdf}
%\includepdf{Contents/00-titlepage.pdf}

% % % Tables of contents, figures, tables
\setcounter{tocdepth}{2}
\pagenumbering{roman}
\tableofcontents %Contents
\thispagestyle{plain}
\listoffigures %Figures
\listoftables %Tables
\newpage %Abstract
\thispagestyle{empty}
%\fancyhf{}
%\thispagestyle{fancy}
%\fancyfoot{\page}
%\thispagestyle{myheadings}
%\renewcommand{\headheight}{0pt}
\phantomsection{\Huge\textbf{Abstract}}\label{s:00-abstract}
%\phantomsection{Abstract}\label{s:00-abstract}
\input{Contents/00-abstract.tex}
\newpage %Abbreviations
\thispagestyle{empty}
%\thispagestyle{fancy}
%\footer{}{Page \thepage of \pageref{lastpage}}
%\cfoot{\page}
\phantomsection{\Huge\textbf{Abbreviations}}\label{s:00-abbrreviations}
%\phantomsection{Abbreviations}\label{s:00-abbreviations}
\thispagestyle{empty}
\input{Contents/00-abbreviations.tex}
\newpage %Symbols
\thispagestyle{empty}
\phantomsection{\Huge\textbf{Symbols}}\label{s:00-symbols}
%\phantomsection{Symbols}\label{s:00-symbols}
\input{Contents/00-symbols.tex}
\newpage %Content
\thispagestyle{empty}
\pagenumbering{arabic}
\setcounter{page}{1}

% % % DISABLED
% % % Different chapter formatting
%\titleformat
%{\chapter} % command
%[display] % shape
%{\bfseries\Large\itshape} % format
%{Story No. \ \thechapter} % label
%{0.5ex} % sep
%{
%	\rule{\textwidth}{1pt}
%	\vspace{1ex}
%	\centering
%} % before-code
%[
%\vspace{-0.5ex}%
%\rule{\textwidth}{0.3pt}
%] % after-code
%
%
%\titleformat{\section}[wrap]
%{\normalfont\bfseries}
%{\thesection.}{0.5em}{}
%
%\titlespacing{\section}{12pc}{1.5ex plus .1ex minus .2ex}{1pc}
% % % DISABLED

% % % TODO



% % % TODO

% % % Content
\chapter{Introduction}\label{c:10-introduction}
\begin{Problem}[Introduction]
\section{Introduction}\label{s:1010-introduction}
\input{Contents/1010-introduction.tex}


\section{Context of the work}\label{s:1020-context-of-the-work}
\indent\indent
\input{Contents/1020-context-of-the-work.tex}


\section{Motivation}\label{s:1030-motivation}
\indent\indent
\input{Contents/1030-motivation.tex}


\section{Problem formulation}\label{s:1040-problem-formulation}
\indent\indent
\input{Contents/1040-problem-formulation.tex}


\section{Contribution and innovation}\label{1050-contribution-and-innovation}
\indent\indent
\input{Contents/1050-contribution-and-innovation.tex}

\end{Problem}

\begin{Problem}[Background theory]
\chapter{Background theory}\label{c:20-background-theory}
%\chapter{Background theory}\label{}
\section{Definitions}\label{s:2010-definitions}
\indent\indent
\input{Contents/2010-definitions.tex}
\input{Contents/2010-definitions-appendix.tex}


\section{Simultaneous localization and mapping}\label{s:2020-simultaneous-localization-and-mapping}
\indent\indent
\input{Contents/2020-simultaneous-localization-and-mapping.tex}

\subsection{Sensors}\label{s:202010-sensors}
\indent\indent
\input{Contents/202010-sensors.tex}

\subsection{Architecture}\label{s:202020-architecture}
\indent\indent
\input{Contents/202020-architecture.tex}

\subsection{Representations}\label{s:202030-representations}
\indent\indent
\input{Contents/202030-representations.tex}


\section{Visual SLAM}\label{s:2030-visual-slam}
\indent\indent
\input{Contents/2030-visual-slam.tex}

\subsection{Monocular}\label{s:203010-monocular}
\indent\indent
\input{Contents/203010-monocular.tex}

\subsection{Stereo vision}\label{s:203020-stereo-vision}
\indent\indent
\input{Contents/203020-stereo-vision.tex}


\section{Collaborative SLAM}\label{s:2040-collaborative-slam}
\indent\indent
\input{Contents/2040-collaborative-slam.tex}


%DECIDE if to include or not, optionally rename
%Basically synonymes, but collaborative refers to a broader spectrum, where multi slam can refer to multiple cameras at the same time;
%\section{Multi SLAM}\label{s:2040-multi-slam}
%\indent\indent
%\input{Contents/2040-multi-slam.tex}


\section{Data fusion}\label{s:2050-data-fusion}
\indent\indent
\input{Contents/2050-data-fusion.tex}


%DECIDE if to include or not
%Relevant rotation invariant features; Review of features with advantages and disadvantages
%\section{Computer vision}\label{s:2055-computer-vision}
%\indent\indent
%\input{Contents/2055-computer-vision.tex}
%
%
%Significant for SfM to a degree, worth mentioning Bogdan's camera model
%\section{Camera models}\label{s:2065-camera-models}
%\indent\indent
%\input{Contents/2065-camera-models.tex}


%\section{Structure from motion}\label{s:2060-structure-from-motion}
%\indent\indent
%\input{Contents/2060-structure-from-motion.tex}
\section{Active perception}\label{s:2060-active perception}
\indent\indent
\input{Contents/2060-active-perception.tex}

\subsection{Structure from motion}\label{s:206010-structure-from-motion}
\indent\indent
\input{Contents/206010-structure-from-motion.tex}


\section{Visual servoing}\label{s:2070-visual-servoing}
\indent\indent
\input{Contents/2070-visual-servoing.tex}


\section{Exploration and rendezvous tasks}\label{s:2080-exploration-and-rendezvouz-tasks}
\indent\indent
\input{Contents/2080-exploration-and-rendevouz-tasks.tex}

\end{Problem}


\begin{Problem}[State of the art]
\chapter{State of the art}\label{c:30-state-of-the-art} %Related work?
Overview of current state of the art:
131016/Cesar Cadena, Luca Carlone, Henry Carrillo, Yasir Latif, Davide Scaramuzza, Jos´e Neira, Ian D. Reid, John J. Leonard - Past, Present, and Future of Simultaneous Localization And Mapping: Towards the Robust-Perception Age.pdf
%ADD another two reviews found from 2015
Add the other review


\section{Simultaneous localization and mapping}\label{s:3010-simultaneous-localization-and-mapping}
\indent\indent
\input{Contents/3010-simultaneous-localization-and-mapping.tex}

\subsection{Standards in SLAM architecture}\label{s:301010-standards-in-slam-architecture}
\indent\indent
\input{Contents/301010-standards-in-slam-architecture.tex}

\subsection{Representations}\label{s:301020-representations}
\indent\indent
\input{Contents/301030-representations.tex}

\subsection{Solutions}\label{s:301035-solutions}
\indent\indent
\input{Contents/301035-solutions.tex}

\subsection{Collaborative SLAM}\label{s:301040-collaborative-slam}
\indent\indent
\input{Contents/301040-collaborative-slam.tex}

\subsection{Open problems}\label{s:301099-open-problems}
\indent\indent
\input{Contents/301099-open-problems.tex}


%DEACTIVATE these placeholders BELOW
\section{Visual servoing}\label{s:3020-visual-servoing}
\indent\indent
\input{Contents/3020-visual-servoing.tex}

\subsection{Visual communication}\label{s:301030-visual-communication}
\indent\indent
\input{Contents/302010-visual-communication.tex}


%\section{Motion planning}\label{s:3030-motion-planning}
%\indent\indent
%\input{Contents/3030-motion-planning.tex}

%\subsection{}\label{s:303010-}
%\indent\indent
%\input{Contents/3030-.tex}


%\section{Control}\label{s:3040-control}
%\indent\indent
%\input{Contents/3040-control.tex}

%\subsection{}\label{s:304010-}
%\indent\indent
%\input{Contents/3040-.tex}


%\section{Optimization}\label{s:3050-optimization}
%\indent\indent
%\input{Contents/3050-optimization.tex}

%\subsection{}\label{s:305010-}
%\indent\indent
%\input{Contents/3050-.tex}
%DEACTIVATE these placeholders ABOVE
\end{Problem}


\begin{Problem}[Collaborative SLAM]
\chapter{Collaborative SLAM}\label{c:40-collaborative-slam}
Previous work of Nived and Nicole as basis
	"Collaborative Visual SLAM Framework for a Multi-Robot System"
	Propose a decentralized approach on that
		Min 7 DOF (position, orientation, scale)
			Other DOFs?
	Introduce visual communication in aspect of rendezvous

%DECIDE how much of this should be in bibliography, it would help making the workplan
\section{Problem statement}\label{s:4010-problem-statement}
\indent\indent
\input{Contents/4010-problem-statement.tex}

\subsection{Assumptions}\label{s:401010-assumptions}
\indent\indent
\input{Contents/401010-assumptions.tex}


\section{Scenarios}\label{s:4020-scenarios}
\indent\indent
\input{Contents/4020-scenarios.tex}


\section{Map characteristics}\label{s:4030-map-characteristics}
\indent\indent
\input{Contents/4030-map-characteristics.tex}


\section{Algorithm}\label{s:4040-algorithm}
\indent\indent
\input{Contents/4040-algorithm.tex}


\end{Problem}


%
%\begin{Problem}[Experimental setup]
%\chapter{Experimental setup}\label{}
\begin{Problem}[Proposed work]
\chapter{Proposed work}\label{c:90-proposed-work}
%GET information about the used hardware and describe in detail what kind of sensors are being used
\section{Experimental setup}\label{s:9010-experimental-setup}
\indent\indent
\input{Contents/9010-experimental-setup.tex}

%\newpage %Workplan
\section{Workplan}\label{s:9090-workplan}
\indent\indent
\input{Contents/9090-workplan.tex}
\end{Problem}


\clearpage
\begin{Problem}[Conclusions]
\section{Conclusions}\label{s:999-conclusions}
\input{Contents/999-conclusions.tex}
\end{Problem}


\clearpage
\begin{Problem}[Bibliography]
%	\section{Bibliography}\label{}
%	\addbibresource{Library.bib}
	\bibliography{Library.bib}
	\bibliographystyle{plainnat}
	\bibliographystyle{IEEEtranS}
%	\bibliography{doc}
	\addcontentsline{toc}{chapter}{Bibliography}
%	\input{Contents/999-bibliography.tex}
\end{Problem}


\end{spacing}
\end{document}


% % % Licence

%----------------------------------------------------------------------%
% The following is copyright and licensing information for
% redistribution of this LaTeX source code; it also includes a liability
% statement. If this source code is not being redistributed to others,
% it may be omitted. It has no effect on the function of the above code.
%----------------------------------------------------------------------%
% Copyright (c) 2007, 2008, 2009, 2010, 2011 by Theodore P. Pavlic
%
% Unless otherwise expressly stated, this work is licensed under the
% Creative Commons Attribution-Noncommercial 3.0 United States License. To
% view a copy of this license, visit
% http://creativecommons.org/licenses/by-nc/3.0/us/ or send a letter to
% Creative Commons, 171 Second Street, Suite 300, San Francisco,
% California, 94105, USA.
%
% % THE SOFTWARE IS PROVIDED "AS IS", WITHOUT WARRANTY OF ANY KIND, EXPRESS
% % OR IMPLIED, INCLUDING BUT NOT LIMITED TO THE WARRANTIES OF
% % MERCHANTABILITY, FITNESS FOR A PARTICULAR PURPOSE AND NONINFRINGEMENT.
% % IN NO EVENT SHALL THE AUTHORS OR COPYRIGHT HOLDERS BE LIABLE FOR ANY
% % CLAIM, DAMAGES OR OTHER LIABILITY, WHETHER IN AN ACTION OF CONTRACT,
% % TORT OR OTHERWISE, ARISING FROM, OUT OF OR IN CONNECTION WITH THE
% % SOFTWARE OR THE USE OR OTHER DEALINGS IN THE SOFTWARE.
%----------------------------------------------------------------------%
